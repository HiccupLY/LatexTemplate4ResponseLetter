% An article class for author response to reviewer comment.
% Copyright (C) 2017 Martin Schroen
% Modifications Copyright (C) 2022 Alex Liu
%
% This program is free software: you can redistribute it and/or modify
% it under the terms of the GNU General Public License as published by
% the Free Software Foundation, either version 3 of the License, or
% (at your option) any later version.
%
% This program is distributed in the hope that it will be useful,
% but WITHOUT ANY WARRANTY; without even the implied warranty of
% MERCHANTABILITY or FITNESS FOR A PARTICULAR PURPOSE.  See the
% GNU General Public License for more details.
%
% You should have received a copy of the GNU General Public License
% along with this program.  If not, see <http://www.gnu.org/licenses/>.



%in the commands, several function are listed as follows:
%\reviewersection : define the new reviewer environment with index added.
%\totalcomment : define the initial comment.
%\generalresponse : define the reply of the initial comment.(option)
%\comment : define the new argue comment of the specific reviewer(\reviewersection).
%\reply : define the reply part.

%\textupdate : define the updated text
%\textdel : define the deleted text


\documentclass{ar2rc}
% \usepackage[UTF8]{ctex}
\usepackage{commands}
\usepackage[english]{babel}
\usepackage{cite}
\usepackage{enumerate}
\usepackage{xspace}


\providecommand {\ly}[1]{\textcolor{orange}{{#1}}}
\providecommand {\todonote}[1]{{\color{ForestGreen}{[TODO:{#1}]}\xspace}}


\title{Your Paper}
\author{Your Name}
\journal{the journal name}
\lognum{12345}

\begin{document}

\maketitle

~\newline

Dear Reviewers and Editors,

Note that the comments are \textit{\textbf{italic}} and the responses are \textbf{regular}. 
% In the manuscript with marked changes, the colour of revised parts is \textupdate{\textbf{blue}}.
In the manuscript with marked changes, the colour of revised parts is \textupdate{\textbf{blue}} and the colour of deleted parts is \textdel{\textbf{red}}.



Sincerely yours,

Name,Address,Email


% Dear Editor,
% We have uploaded three files: % 这次上传的三个文件包括
% \begin{enumerate}[A]
%     \item our comment-by-comment response to the questions (below) (response to editor);
%     \item the updated PDF;
%     \item the updated latex archive.
% \end{enumerate}
% Best regards,

% Yang Liu, Zheng Zheng, Fangyun Qinb, Xiaoyi Zhang, Haonan Yao
\newpage


\section*{Main Revisions in the New Submission}
% General intro text goes here

\lipsum[2-3]


\newpage



\section*{Responses to the Reviewers’ Comments}

\reviewersection
\label{reviewer.1}




% comment one description 
\begin{comment} \label{pt:1.1}
    \lipsum[1-1]

\end{comment}

% Our response
\begin{response}
    \lipsum[1-1]
\end{response}

\begin{comment} \label{pt:1.2}
    \lipsum[1-2]
\end{comment}

\begin{response}
	\ly{\lipsum[1-1]}

    \todonote{not fixed}
\end{response}



% Begin a new reviewer section
\reviewersection



\begin{comment}
    \lipsum[1-1]

\end{comment}


\begin{response}
    To address this phenomenon, we have added the relevant description as follows.

    \begin{quote}
        \textupdate{
            This is a demo.        
            This is a demo.
            This is a demo.
            This is a demo.
            This is a demo.
            This is a demo.
            This is a demo.
        }
        \textdel{
        This is a demo.        
        This is a demo.
        This is a demo.
        This is a demo.
        This is a demo.
        This is a demo.
        This is a demo.
        }
        
        The cat in the box is \DIFdelbegin \DIFdel{dead}\DIFdelend \DIFaddbegin \DIFadd{alive}\DIFaddend .
        \begin{align}
        E &= mc^2 \\
        m\cdot \DIFdelbegin \DIFdel{a=F}\DIFdelend \DIFaddbegin \DIFadd{v=p}\DIFaddend .
        \end{align}
        OK, a citation\cite{BADUE2021113816,chen2018unmanned}.
    \end{quote}
\end{response}



\bibliographystyle{unsrt}
\bibliography{mybibfile}

\end{document}
